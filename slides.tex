\documentclass{beamer}
\usepackage{listings}
\setbeamersize{text margin left=2pt,text margin right=2pt}

\lstset
{
  language     = C++,
  basicstyle=\ttfamily,
  columns      = flexible,
  literate     = {-}{{\Processdash}}1,
  keywordstyle=\color{blue},
  keepspaces,
}

%[language=C++,basicstyle=\ttfamily,keywordstyle=\color{red}]

\makeatletter
\newcommand\Processdash%
{%
    \lst@ifLmode% if in one-line comment mode
        -%
    \else%
        $-$%
    \fi%
}
\makeatother

\begin{document}
\title{Implementation of std::rotate algorithm}
\author{Aditya Kumar}
\date{Aug. 3, 2016}
%\date{\today}

\frame{\titlepage}

\frame{\frametitle{std::rotate(Introduction)}
  \begin{itemize}
  \item What is std::rotate 
  \item Uses of std::rotate
  \end{itemize}
}

\defverbatim[colored]\lstH{
  \begin{lstlisting}
    // I models Forward Iterator
    template<class I> // (until C++11)
    void rotate(I first, I n_first, I last);

    template<class I> // (since C++11)
    I rotate(I first, I n_first, I last);

    Performs a left rotation on a range of elements.

    vector<int> s{0, 1, 2, 2, 3, 4, 5, 7, 7, 10};
    rotate(s.begin(), s.begin()+1, s.end());
    // rotate left : 1, 2, 2, 3, 4, 5, 7, 7, 10, 0
  \end{lstlisting}
}

\begin{frame}{What is std::rotate}
  \lstH
\end{frame}

\defverbatim[colored]\lstI{
  \begin{lstlisting}
    // I models Input Iterator to T
    template <typename T, typename I>
    void insert(vector<T>& v,
                typename vector<T>::iterator ip,
                I first, I limit) {
      auto n = v.end() - v.begin();
      while (first != limit) {
        v.push_back(*first);
        ++first;
      }
      rotate(ip, v.begin() + n, v.end());
    }
  \end{lstlisting}
}

\begin{frame}{Uses: Inserting elements in vector using rotate in O(n)}
  \lstI
\end{frame}

\defverbatim[colored]\lstI{
  \begin{lstlisting}
    // It models Random Access Iterator
    template <typename It>
    auto slide(It f, It l, It p) -> std::pair<It, It> {
      if (p < f) return { p, std::rotate(p, f, l) };
      if (l < p) return { std::rotate(f, l, p), p };
      return { f, l };
    }
  \end{lstlisting}
}

\begin{frame}{Uses: Sliding a continuous set of elements in a container to another position}
  \lstI
\end{frame}

\defverbatim[colored]\lstJ{
  \begin{lstlisting}
    // I models Bidirectional Iterator
    // f: first
    // m: rotation point
    // l: limit
    // [f, l) is valid and m is in [f, l)
    template <typename I>
    void rotate(I f, I m, I l) {
      reverse(f, m);
      reverse(m, l);
      reverse(f, l);
    }
  \end{lstlisting}
}

\frame{\frametitle{Implementation using bidirectional iterator}
\lstJ
}

\defverbatim[colored]\lstK{
  \begin{lstlisting}
    // I models Forward Iterator
    // f: first
    // m: rotation point
    // l: limit
    // [f, l) is valid and m is in [f, l)
    template <typename I>
    void rotate(I f, I m, I l) {
      pair<I, I> p = swap_ranges(f, m, m, l);
      I u = p.first;
      I v = p.second;
      if (v != l)
        rotate(u, v, l);
      else if (u != m)
        rotate(u, m, l);
    }
  \end{lstlisting}
}

\frame{\frametitle{Implementation using forward iterator}
\lstK
}

\defverbatim[colored]\lstL{
  \begin{lstlisting}
    // I models Forward Iterator
    // f: first
    // m: rotation point
    // l: limit
    // [f, l) is valid and m is in [f, l)
    template <typename I>
    pair<I, I> rotate(I f, I m, I l) {
      DISTANCE_TYPE(I) n = gcd(m - f, l - m);
      rotate_iterator_action<I> action(f, m, l);
      while (n > 0) {
        --n;
        do_cycle(f + n, action);
      }
      I n_m = f + (l - m);
      return (n_m < m) ? pair<I, I>(n_m, m)
                       : pair<I, I>(m, n_m);
    }
  \end{lstlisting}
}

\frame{\frametitle{Implementation using random-access iterator}
  \lstL
}

\end{document}
